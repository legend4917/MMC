
\geometry{a4paper}                   % ... or a4paper or a5paper or ... 

% 全文控制
\geometry{left=2.5cm,right=2.5cm,top=2.5cm,bottom=3.5cm}		% 正文与纸张边距
\setlength{\parindent}{24pt}	% 段首行缩进
\linespread{1.5}\selectfont		% 行距设置
%\linespread{1.5}
%\renewcommand{\baselinestretch}{1.38}
% \setlength{\parskip}{1ex}
%\setlength{\parskip}{0.5\baselineskip}
\pagestyle{fancy}
\fancyhead{}
\chead{\xiaowu 河海大学本科毕业论文}  
\setlength{\topmargin}{-0.5cm}
\setlength{\headsep}{0.8cm}

%\setmathfont[math-style=ISO,bold-style=ISO,vargreek-shape=TeX]{TeX Gyre Termes Math}

\setmainfont{Times New Roman}	%缺省英文字体
\defaultfontfeatures{Mapping=tex-text}
\setCJKmainfont{STSong}



\setCJKfamilyfont{heiti}{STHeitiSC-Medium}                                    %黑体  heiti
\newcommand{\heiti}{\CJKfamily{heiti}}                          % 黑体\newcommand{\chuhao}{\fontsize{42pt}
\setCJKfamilyfont{songti}{STSong}                                    %宋体  songti 
\newcommand{\songti}{\CJKfamily{songti}} 
\setCJKfamilyfont{fangsong}{STFangsong} 
\newcommand{\fangsong}{\CJKfamily{fangsong}} 

\newcommand{\chuhao}{\fontsize{42pt}{46pt}\selectfont}
\newcommand{\xiaochuhao}{\fontsize{36pt}{40pt}\selectfont}
\newcommand{\yichu}{\fontsize{32pt}{36pt}\selectfont}
\newcommand{\yihao}{\fontsize{26pt}{32pt}\selectfont}
\newcommand{\erhao}{\fontsize{21pt}{24pt}\selectfont}
\newcommand{\xiaoer}{\fontsize{18pt}{20}\selectfont}
\newcommand{\sanhao}{\fontsize{15.75pt}{18pt}\selectfont}
\newcommand{\xiaosan}{\fontsize{15pt}{22.5pt}\selectfont}
\newcommand{\sihao}{\fontsize{14pt}{21pt}\selectfont}
\newcommand{\xiaosi}{\fontsize{12pt}{18pt}\selectfont}
\newcommand{\wuhao}{\fontsize{10.5pt}{13pt}\selectfont}
\newcommand{\xiaowu}{\fontsize{9pt}{11pt}\selectfont}
\newcommand{\liuhao}{\fontsize{7.5pt}{9pt}\selectfont}
\newcommand{\xiaoliuhao}{\fontsize{6.5pt}{7.5pt}\selectfont}
\newcommand{\qihao}{\fontsize{5.5pt}{6.5pt}\selectfont}


\setlength{\parskip}{0pt}% 段距
\newcommand{\mainlineskip}{1.3}% 主行距1.3
\newcommand{\linespacing}[1]{\linespread{#1}\selectfont}% 行距命令

\renewcommand{\contentsname}{\heiti\xiaoer{目\hspace{1em}录}}
\renewcommand\bibname{参考文献}
\renewcommand\thefigure{\thechapter.\arabic{figure}}
\renewcommand{\figurename}{图}
\renewcommand{\theequation}{\arabic{chapter}.\arabic{equation}}
\newcommand{\upcite}[1]{\textsuperscript{\textsuperscript{\cite{#1}}}}






%\DeclareSymbolFont{operators}   {OT1}{ztmcm}{m}{n}
%\DeclareSymbolFont{symbols}     {OMS}{ztmcm}{m}{n}
%\DeclareSymbolFont{bold}        {OT1}{ptm}{bx}{n}
%\DeclareSymbolFont{italic}      {OT1}{ptm}{m}{it}
%\DeclareMathSymbol{\alpha}{0}{letters}{"0B}

%\DeclareSymbolFontAlphabet{\mathrm}{letters}



% 定义目录格式
\titlecontents{chapter}[0pt]{\bfseries\sihao}{第\CJKnumber{\thecontentslabel}章\quad}{}
        {\hspace{0.3em}\titlerule*[4pt]{$\cdot$}\contentspage}
        
\titlecontents{section}[0pt]{\songti\xiaosi}{\thecontentslabel}{}
        {\hspace{0.3em}\titlerule*[4pt]{$\cdot$}\contentspage}

\titlecontents{subsection}[0pt]{\songti\xiaosi}{\thecontentslabel}{}
        {\hspace{0.3em}\titlerule*[4pt]{$\cdot$}\contentspage}
              
% 定义章节标题格式
\titleformat{\chapter}[hang]{\centering\xiaoer\heiti}{第\CJKnumber{\thechapter}章}{1em}{}
\titlespacing{\chapter}{0pt}{-3ex  plus .1ex minus .2ex}{3.3ex}

\titleformat{\section}{\sihao\heiti}{\thesection}{1em}{}
\titlespacing{\section}{0pt}{0.5em}{0.5em}

\titleformat{\subsection}{\xiaosi\heiti}{\thesubsection}{1em}{}
\titlespacing{\subsection}{0pt}{0.5em}{0.3em}

\titleformat{\subsubsection}{\xiaosi\heiti}{\thesubsubsection}{1em}{}
\titlespacing{\subsubsection}{0pt}{0.3em}{0pt}

% 定义列表环境
\setlength{\itemindent}{24pt}    %标签缩进量
\setenumerate[1]{itemsep=0pt,partopsep=0pt,parsep=\parskip,topsep=0pt}
\setitemize[1]{itemsep=0pt,partopsep=0pt,parsep=\parskip,topsep=5pt}
\setdescription{itemsep=0pt,partopsep=0pt,parsep=\parskip,topsep=5pt}

\setlength{\bibsep}{0.5ex}  % vertical spacing between references

%\punctstyle{kaiming} 

\renewcommand*{\arraystretch}{0}		% 调整矩阵行间距



\renewcommand{\figurename}{图}
\renewcommand{\tablename}{表}
\renewcommand{\thetable}{\thechapter-\arabic{table}} %修改表格的表头为表2-1
\setlength{\belowcaptionskip}{5pt}

\renewcommand\arraystretch{1}

\theoremstyle{plain} 
\newtheorem{theorem}{\heiti{\hskip 2em 定理}}[chapter]
\setlength{\theorempreskipamount}{6pt}
\setlength{\theorempostskipamount}{-20pt}
