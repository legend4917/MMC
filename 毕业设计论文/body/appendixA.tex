\chapter*{附录A}
\addcontentsline{toc}{chapter}{附录A}
{\CJKfontspec{STHeitiSC-Medium}证明}\hspace{1em}首先,问题(\ref{equ:MMC-CP})与下列问题等价:
\begin{equation*}
\begin{split}
\min_{M.\delta} \quad & \delta \\
s.t. \quad & \delta \le \max_{\mathbf{\alpha}}2\mathbf{\alpha}^T\mathbf{e}-\langle K \circ \mathbf{\alpha}\mathbf{\alpha}^T,M \rangle,0 \le \mathbf{\alpha} \le C, \mathcal{L}_1,\mathcal{L}_2,\mathcal{L}_4,M \succeq 0
\end{split}
\end{equation*}

令$G(K)=M \circ K$,上述问题的拉格朗日函数为:
\begin{equation*}
\begin{split}
L(\mathbf{\alpha},\mathbf{\mu},\mathbf{\nu}) = 2\mathbf{\alpha}^T\mathbf{e} - \mathbf{\alpha}^TG(K)\mathbf{\alpha} + 2\mathbf{\mu}\mathbf{\alpha} + 2\mathbf{\nu}(C-\mathbf{\alpha})
\end{split}
\end{equation*}

拉格朗如函数$L$对$\mathbf{\alpha}$求偏导可得:
\begin{equation*}
\begin{split}
\frac{\partial L}{\partial\mathbf{\alpha}}=0 \Longrightarrow \mathbf{\alpha} = G(K)^{-1}(\mathbf{e} + \mathbf{\mu} - \mathbf{\nu}) 
\end{split}
\end{equation*}

将上式代入拉格朗日函数可得
\begin{equation*}
\begin{split}
W(\mathbf{\mu},\mathbf{\nu}) & = \max_{\mathbf{\alpha}}\min_{\mathbf{\mu}\ge 0,\mathbf{\nu}\ge 0}L \\
& = \min_{\mathbf{\mu}\ge 0,\mathbf{\nu}\ge 0}\max_{\mathbf{\alpha}} L \\
& = \min_{\mathbf{\mu}\ge 0,\mathbf{\nu}\ge 0}(\mathbf{e}+\mathbf{\mu}-\mathbf{\nu})^TG(K)^{-1}(\mathbf{e}+\mathbf{\mu}-\mathbf{\nu}) + 2C\mathbf{v}^T\mathbf{e}
\end{split}
\end{equation*}

那么要使得$W(\mathbf{\mu},\mathbf{\nu}) \le \delta$,必存在$\mathbf{\mu}\ge 0,\mathbf{\nu}\ge 0$,使得:
$$(e+\mathbf{\mu}-\mathbf{\nu})^TG(K)^{-1}(\mathbf{e}+\mathbf{\mu}-\mathbf{\nu}) + 2C\mathbf{v}^T\mathbf{e} \le \delta$$

由Schur补引理即可得到等价的问题(\ref{equ:MMC-CP})。
