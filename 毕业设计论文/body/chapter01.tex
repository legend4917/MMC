\chapter{绪论}

\pagenumbering{arabic}
\section{研究背景与意义}

聚类是一种无监督的学习,其训练样本的标记信息是未知的,目标是通过对无标记训练样本的学习来揭示数据的内在性质规律,为进一步的数据分析提供基础。聚类试图将数据集中的样本划分为若干个通常是不想交的子集,每个子集称为一个“簇”,使得簇内样本差异很小,簇间样本差异较大。聚类既可以作为一个单独过程,用于寻找数据内在的分布结构,也可作为分类等其他学习任务的前驱过程\upcite{machinelearning}。

聚类是机器学习和数据挖掘领域十分重要的研究课题,尤其在数据挖掘中的应用极为广泛。在商务上,聚类能帮助市场分析人员从客户基本库中发现不同的客户群,并且用购买模式来刻画不同的客户群的特征等;在生物学上,聚类能用于推导植物和动物的分类,对基因进行分类,获得对种群中固有结构的认识等等;在搜索引擎上,聚类能对web上的文档进行分类,挖掘信息等。

但是,传统的聚类只能解决低维数据的聚类问题。由于实际应用中数据十分复杂,比如基因表达数据、图像特征、文档词频数等,其维度能达到成百上千,甚至更高。传统的聚类方法在高维数据集中进行聚类时,主要遇到两个问题:
  \begin{enumerate}[fullwidth,itemindent=24pt]
  \item 高维数据集中存在大量无关的属性,使得在所有维中存在簇的可能性几乎为零;
  \item 高维空间中数据较低维空间中数据分布要稀疏,其中数据间距离几乎相等是普遍现象,而传统聚类方法是基于距离进行聚类的,因此在高维空间中无法基于距离来构建簇。
  \end{enumerate}

\section{聚类方法研究现状}
当前聚类算法主要包括原型聚类、密度聚类和层次聚类等。原型聚类中最简单的当属k均值算法,该算法存在大量的变体,比如k-medoids算法\upcite{kaufman1987clustering}强制原型向量必为训练样本,k-modes算法\upcite{huang1998extensions}可处理离散属性,Fuzzy C-means (FCM) \upcite{bezdek2013pattern}则是“软聚类”算法,允许每个样本以不同程度同时属于多个原型。需要注意的是,k均值类算仅在凸形簇结构上效果很好。最近研究表明,若采用某种Bregman距离,则可显著增强此类算法对更多类型簇结构的适用性\upcite{banerjee2005clustering}。引入核技巧则可得到核k均值算法\upcite{scholkopf1998nonlinear},这与谱聚类\upcite{von2007tutorial}有模切联系。谱聚类可以看作在拉普拉斯特征映射降维后执行k均值聚类。聚类簇数k通常需要用户提供,有一些启发式用于自动确定k\upcite{pelleg2000x},但常用的仍然是基于不同k值多次运行后选取最佳结果。

采用不同方式表征样本分布的紧密程度,可设计出不同的密度聚类算法,除DBSCAN\upcite{ester1996density}外,较常用的还有OPTICS\upcite{ankerst1999optics}、DENCLUE\upcite{hinneburg1998efficient}等。AGNES\upcite{kaufman2009finding}采用自底向上的聚类策略来产生层次聚类结构,与之相反,DIANA\upcite{kaufman2009finding}则采用自顶向下的分拆策略。AGNES和DIANA都不能对已合并或已分拆的聚类簇进行回溯调整,常用的层次聚类如BIRCH\upcite{zhang1996birch}、ROCK\upcite{guha1999rock}等对此进行了改进。

\section{本文的研究动机}
支持向量机以间隔最大化为学习策略,并引入核技巧实现非线性分类器。其优越的分类精度,尤其在文本分类任务中显示出卓越性能\upcite{joachims2000text},被业界公认为最好的分类模型。间隔最大化能保证模型具有较好的泛化能力,核技巧能有效避免“维数灾难”,考虑到间隔最大化和核技巧在分类的卓越性能,那么,将其应用在聚类技术中,取得的性能让人十分期待。

\section{研究目标及内容}
本文的研究目标是:基于间隔最大化学习策略和核技巧的思想,着重研究最大间隔聚类\upcite{mmc} (MMC) 算法和多核聚类\upcite{mkc} (MKC) 算法,详细探讨间隔最大化和核技巧在聚类中所取得的性能。


本文的主要工作内容:
\begin{enumerate}[fullwidth,itemindent=24pt]
  \item 分析间隔最大化学习策略和核技巧的优越之处。从SVM主问题出发以及对偶问题出发研究间隔最大化学习策略和核技巧在SVM中的应用方法。
  \item 分析最大间隔聚类算法原理和模型推导过程。针对分类模型SVM,将其推广到聚类分析中,并通过凸优化技巧进行建模。
  \item 分析多核聚类算法原理和模型推导过程。针对最大间隔聚类算法的局限性,对其进行改进,由单核学习到多核学习。并引入割平面算法和凹凸规划,提高算法的收敛速度。
  \item 针对最大间隔聚类算法和多核聚类算法,使用UCI上的数据集,通过实验来验证算法的性能,并与k均值聚类和谱聚类等算法进行对比。
\end{enumerate}
  
\section{论文结构}
本文共分为六章。第一章介绍了本文工作的研究背景与意义,聚类的研究现状,研究动机以及主要的研究内容和目标;第二章介绍了SVM模型的基本内容,分别从SVM主问题以及对偶问题出发,分析SVM中的间隔最大化学习策略和核技巧的应用,为后续章节提供必要的背景知识;第三章分析了MMC算法的模型推导过程,并提出该算法的局限性;第四章分析了MKC算法的模型推导过程,并提出该算法的优越之处;第五章进行相关实验来验证MMC算法和MKC算法的性能,并与传统的聚类算法进行对比;第六章对本文的工作进行了总结。