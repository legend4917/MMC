\chapter{绪论}

\pagenumbering{arabic}
\section{研究背景与意义}

聚类是一种无监督的学习,其训练样本的标记信息是未知的,目标是通过对无标记训练样本的学习来揭示数据的内在性质规律,为进一步的数据分析提供基础。聚类试图将数据集中的样本划分为若干个通常是不想交的子集,每个子集称为一个“簇”,使得簇内样本差异很小,簇间样本差异较大。聚类不但能作为一个单独的过程,用于寻找样本数据内部的分布情况,也可以作为分类等其他学习任务的前驱过程\upcite{machinelearning}。

在数据挖掘和机器学习等领域,聚类是十分重要的研究课题。尤其是在数据挖掘领域,聚类的应用极为广泛。在电子商务上,聚类能够将具有相似浏览行为的客户进行分组,并分析客户相同的行为特征,能够更好的帮助工作人员了解自己的客户,以便向客户提供更加精确的服务;在生物学上,聚类能对动植物的基因进行分类,分析种群固有的结构知识等等;在搜索引擎上,聚类能对网页和文档等进行分类,挖掘其中的信息等;尤其是在社交网络挖掘等领域,聚类技术的应用极为广泛。

但是,传统的聚类方法只能应用于一些维度较低的数据。由于在实际应用中,数据往往十分复杂,维度较高,比如基因表达数据、图像特征、文档词频数等,其维度通常能达到成百上千,甚至更高。在高维数据集中进行聚类时,传统的聚类方法主要遇到两个问题:
  \begin{enumerate}[fullwidth,itemindent=24pt]
  \item 高维数据集中存在许多与类别无关的特征,这就使得在所有特征维度下,存在簇的可能性非常低;
  \item 在高维空间中,数据的分布会比低维空间更加稀疏,这造成各个数据点之间的距离几乎相等,差距不大。而传统的聚类方法基本上都是以数据点之间的距离作为相似度的衡量标准,因此很难在高维空间中进行聚类。
  \end{enumerate}
  
在传统的聚类技术很难解决高维数据集聚类问题时,核技巧在有监督的分类学习任务中展现出它的优越的性能,其中应用最为广泛的分类模型就是支持向量机SVM。得益于核技巧的应用,以间隔最大化为学习策略的SVM只依赖于由样本数据生成的核矩阵,从而有效解决高维数据集问题,并且很容易的实现了非线性分类。

受到SVM卓越的分类性能的启发,将间隔最大化学习策略和核技巧从有监督的分类学习任务推广到无监督的聚类学习任务中,其学习得到的模型必然会比传统的聚类模型具有更多的优越之处。一方面核技巧能有效解决高维数据集问题,另一方面间隔最大化学习策略保证了模型良好的泛化性能,如何将其应用到无监督的聚类学习任务中,具有非常重大的研究意义。

\section{聚类方法研究现状}
传统的聚类算法主要包括原型聚类、层次聚类以及密度聚类等。原型聚类中最简单的当属k均值算法,该算法存在大量的变体,比如k-medoids算法\upcite{kaufman1987clustering}强制原型向量必为训练样本,k-modes算法\upcite{huang1998extensions}可处理离散属性,Fuzzy C-means (FCM) \upcite{bezdek2013pattern}则是“软聚类”算法,允许每个样本以不同程度同时属于多个原型。需要注意的是,k均值类算仅在凸形簇结构上效果很好。最近研究表明,若采用某种Bregman距离,则可显著增强此类算法对更多类型簇结构的适用性\upcite{banerjee2005clustering}。引入核技巧则可得到核k均值算法\upcite{scholkopf1998nonlinear},这与谱聚类\upcite{von2007tutorial}有模切联系。谱聚类可以看作在拉普拉斯特征映射降维后执行k均值聚类。聚类簇数k通常需要用户提供,有一些启发式用于自动确定k\upcite{pelleg2000x},但常用的仍然是基于不同k值多次运行后选取最佳结果。

采用不同方式表征样本分布的紧密程度,可设计出不同的密度聚类算法,除DBSCAN\upcite{ester1996density}外,较常用的还有OPTICS\upcite{ankerst1999optics}、DENCLUE\upcite{hinneburg1998efficient}等。AGNES\upcite{kaufman2009finding}采用自底向上的聚类策略来产生层次聚类结构,与之相反,DIANA\upcite{kaufman2009finding}则采用自顶向下的分拆策略。AGNES和DIANA都不能对已合并或已分拆的聚类簇进行回溯调整,常用的层次聚类如BIRCH\upcite{zhang1996birch}、ROCK\upcite{guha1999rock}等对此进行了改进。

近些年来核技巧开始逐渐被用在聚类分析中。Tax通过使用支持向量方法来描述数据域,并提出了一种基于高斯核的SVDD (Support Vector Domain Description) 算法\upcite{tax1999support}。Ben-Hen提出了支持向量聚类 (SVC)\upcite{ben2002support}算法。作为一种新的无监督非参数型算法,SVC主要分为两个过程: SVM训练过程和标记分配过程,其中SVC训练过程主要包括确定高斯核宽度系数、计算核矩阵和拉格朗如乘子、选取支持向量并计算在高维特征空间中特征球的半径;标记分配过程首先会生成关联矩阵,然后根据关联矩阵进行标记分配\upcite{吕常魁2005一种支持向量聚类的快速算法}。

就目前的聚类方法而言,大多数都只能在特定的样本数据集上表现出较好的性能,包括传统的k-means聚类算法和模糊C均值聚类算法等等,这些算法都是直接基于样本数据特征进行聚类,却没有对样本数据的特征进行优化,因此上面这些方法的聚类性能往往在很大程度上取决于样本数据内部的分布情况\upcite{anzai2012pattern}。但是,通过使用核技巧,将输入空间的数据映射到高维的特征空间中,增加各类样本数据之间的差异,以达到在高维特征空间中线性可聚的目的,从而提高聚类的精确度\upcite{anzai2012pattern}。随着近些年来对核聚类的探索,涌现出许多基于核的聚类算法。比如支持向量聚类,基于核的模糊聚类算法,基于模糊核聚类的SVM多类分类方法等等。核聚类研究的进展与突破,为有效处理非线性样本数据带来了突破口,同时也拓展了聚类课题的研究范围。

\section{研究目标及内容}
支持向量机以间隔最大化为学习策略,并引入核技巧实现非线性分类器。其优越的分类精度,尤其在文本分类任务中显示出卓越性能\upcite{joachims2000text},被业界公认为最好的分类模型。间隔最大化学习策略能保证模型具有较好的泛化能力,核技巧能轻松实现非线性分类,并且能有效避免“维数灾难”。考虑到间隔最大化学习策略和核技巧在分类学习任务中的卓越性能,那么将其应用在无监督学习的聚类任务中,取得的性能让人十分期待。

本文的研究目标是:基于间隔最大化学习策略和核技巧的思想,着重研究其在最大间隔聚类 (MMC)\upcite{mmc} 算法和多核聚类 (MKC)\upcite{mkc} 算法的应用技巧,详细探讨间隔最大化学习策略和核技巧在聚类中所取得的性能。

本文的主要工作内容:
\begin{enumerate}[fullwidth,itemindent=24pt]
  \item 分析间隔最大化学习策略和核技巧的优越之处。分别从SVM主问题以及对偶问题出发研究间隔最大化学习策略和核技巧在SVM中的应用方法。
  \item 分析最大间隔聚类算法原理和模型推导过程。针对分类模型SVM,将其推广到聚类分析中,并通过凸优化技巧进行建模。
  \item 分析多核聚类算法原理和模型推导过程。针对最大间隔聚类算法的局限性,对其进行改进,由单核学习到多核学习。并引入割平面算法和凹凸规划,提高算法的收敛速度。
  \item 针对最大间隔聚类算法和多核聚类算法,使用UCI上的数据集,通过实验来验证算法的性能,并与k均值聚类和谱聚类等算法进行对比。
\end{enumerate}
  
\section{论文结构}
本文共分为六章。第一章介绍了本文工作的研究背景与意义,聚类的研究现状,研究动机以及主要的研究内容和目标;第二章介绍了SVM模型的基本内容,分别从SVM主问题以及对偶问题出发,分析SVM中的间隔最大化学习策略和核技巧的应用,为后续章节提供必要的背景知识;第三章分析了MMC算法的模型推导过程,并提出该算法的局限性;第四章分析了MKC算法的模型推导过程,并提出该算法的优越之处;第五章进行相关实验来验证MMC算法和MKC算法的性能,并与传统的聚类算法进行对比;第六章对本文的工作进行了总结。