\chapter*{算法源码}
\addcontentsline{toc}{chapter}{算法源码}
\noindent 本文中的MKC算法的核心Matlab代码如下:
\begin{lstlisting}[basicstyle=\small\menlo]
% 参数介绍
% data为待输入的训练数据,矩阵类型,每行表示一条训练数据,每列表示训练数据的属性
% label为训练数据的标签,列向量,元素取值为-1或1
% C为松弛参数
% d为多项式核函数的参数
% sigma为高斯核函数的参数
% epsilon为割平面过程的终止条件,这里默认为0.05
% alpha为CCCP过程的终止条件,这里默认为0.01
% best_v为在当前模型参数下,割平面过程中使得聚类精度最高的权值参数
% best_b为在当前模型参数下,割平面过程中使得聚类精度最高的偏置参数
% max_acc为在当前模型参数下的最高聚类精确度
function [ best_v, best_b, max_acc ] = mkc( data, label, C, d, sigma, epsilon, alpha )
if nargin == 5	% 设置默认值
    epsilon = 0.05;
    alpha = 0.01;
end
[m,n] = size(data);
l = m/3;  	% 类平衡约束条件
M = 3;     % 核函数个数,这里包括线性核函数、多项式核函数、高斯核函数
linear_kernel = data * data';  	% 线性核函数生成核矩阵
polynomial_kernel = (data * data'+1).^d; % 多项式核函数生成核矩阵
guass_kernel = zeros(m,m);
for i = 1:m
    for j = 1:m
	data_temp = data(i,:) - data(j,:);
	guass_kernel(i,j) = exp(-data_temp*data_temp'/(2*sigma)); 	% 高斯核函数生成核矩阵
    end
end
X = {};
X{1} = getFeature(linear_kernel);	% 根据核矩阵获取其特征
X{2} = getFeature(polynomial_kernel);	% 根据核矩阵获取其特征
X{3} = getFeature(guass_kernel);	% 根据核矩阵获取其特征
C_Omega = {};		% 保存所有最违背的约束,也即约束子集Omega
max_acc = -inf;
% 开始割平面过程,在每轮割平面中进行CCCP过程,并求解每轮CCCP中的最优化问题
while 1	% 割平面迭代过程
    v_k = ones(m,M);
    b_k = 1;
    res_temp = inf;
    while 1		% CCCP迭代过程
        cvx_begin sdp quiet		% 使用cvx优化包,一次CCCP求解过程
            cvx_solver mosek		% 使用mosek求解器
            variables betta(M) T(M) 
            variables xi(1) b(1) 
            variable v(m,M)
            minimize(0.5*sum(T)+C*xi)	% 最小化目标函数
            subject to 		% 所有约束条件
                % 基础约束
                for i=1:M
                    norm([2*v(:,i);T(i)-betta(i)]) <= T(i)+betta(i);
                    betta(i) >= 0;
                end
                sum(betta.^2) <= 1;
                xi >= 0;
                l_temp = zeros(1,m);
                for i=1:M
                    l_temp = l_temp + v(:,i)'*X{i}';
                end
                -l <= sum(l_temp+b) <= l; 
                % 由最违背约束子集Omega生成的约束条件
                if numel(C_Omega) > 0
                    z_temp = zeros(1,m);
                    for i=1:M
                        z_temp = z_temp + v_k(:,i)'*X{i}';
                    end
                    z = sign(z_temp+b_k)';
                    for i=1:numel(c_arr)
                        1/m*sum(C_Omega{i}) <= xi+1/m*((C_Omega{i}.*z)'*(l_temp+b)');
                    end
                end
        cvx_end		% 最优化问题求解完毕
        res = cvx_optval;
        % 判断是否满足CCCP迭代终止条件
        if abs((res-res_temp)/res) <= alpha
            break;
        end
        v_k = v;
        b_k = b;
        res_temp = res;
    end	% CCCP迭代终止,进入割平面过程
    temp = zeros(1,m);
    for i=1:M
        temp = temp + v(:,i)' * X{i}';
    end
    temp = (temp + b)';
    c = zeros(m,1);     % 初始化最违背约束对应的c向量
    xi_m = 0;
    for i = 1:m     % 寻找最违背的约束
        if abs(temp(i)) < 1
            c(i) = 1;
        else
            c(i) = 0;
        end
        xi_m = xi_m + c(i) * (1 - abs(temp(i)));
    end
    xi_m = xi_m / m;
    cluster = sign(temp);	% 聚类生成的样本类别标记
    acc = sum(label==cluster)/m;	% 计算聚类准确率
    if acc < 0.5		% 保证多数原则
        acc = 1 - acc;
    end
    if max_acc < acc		% 判断本次迭代的聚类准确率是否最优
        max_acc = acc;
        best_v = v_k;
        best_b = b_k;
    end
    if xi_m-xi <= epsilon	% 判断是否满足割平面过程的终止条件
        break;
    end
    % 将最违背的约束加入约束子集,再进入CCCP过程
    C_Omega{numel(C_Omega)+1} = c;
end
end

% 特征提取函数参数介绍
% K为由核函数生成的样本矩阵,也即核矩阵
% X为对核矩阵进行特征分解得到的特征向量
function [ X ] = getFeature( K )
n = length(K);
[p,x] = eig(K);
for i=1:n
    if x(i,i) < 0
        x(i,i) = 0;
    end
    x(i,i) = x(i,i)^0.5;
end
X = p * x;
end
\end{lstlisting}