\chapter{多核聚类MKC}
经过上一章的分析,MMC的基本原理和模型推导过程都已经十分清晰。但很容易想到,使用半定规划(SDP)解决凸整形优化问题(\ref{equ:MMC-origin})的过程损失了部分参数空间,使得最终求解的最优类标记与实际的类标记之间有一定的偏差。并且MMC与其它核方法一样,核函数的选择将直接决定模型性能的好坏,但目前如何选择合适的核函数仍然是未解决的问题。针对MMC的适用性,在本章中,受到有监督学习中多核学习工作\upcite{mmkl}的鼓舞,在MMC模型的基础上,引入多核学习的思想得到多核聚类(MKC)。MKC针对MMC中的非凸整形优化问题(\ref{equ:MMC-origin}),使用割平面算法\upcite{zhao2008efficient,zhao2008efficient}和凹凸规划进行求解。最终得到最大间隔超平面、最适当的类标记组合以及最优的核函数组合。

\section{MKC模型}
\subsection{多核学习}
MMC与其它核方法一样,都依赖于将数据样本映射到高维的特征空间。但它们都面临一个核心的问题,对于一个特定的任务,不清楚哪个核函数最适合。因此,最近SVM和其它核方法从一系列性质相同或者不同的核中构造出一个核,而不是使用单一固定的核,这样的工作已经显示出令人激动的结果\upcite{mmkl}。允许数据样本在不同特征空间的映射合并为一个基核,这样带来很大的灵活性。基于这些启发,MKC将MMC的单核学习改进为多核学习,详细的说,MKC考虑将$M$个特征映射$\Phi_1,\cdots,\Phi_M$ (对应于$M$个基核$K_1,\cdots,K_M$)进行非负线性组合,公式化描述如下:
\begin{align} % requires amsmath; align* for no eq. number
   \Phi(\mathbf{x})=\sum^M_{k=1}\beta_k\Phi_k(\mathbf{x})
\end{align}

其中,$\beta_k\ge 0$,并且存在整数$p$,使得$\sum_k\beta^p_k \le 1$。通过同时考虑到超平面参数(权值$\mathbf{w}$和偏置$b$)和权重参数$\beta_k$来优化MMC中的目标函数,就能得到最优的MMC特征映射。

\subsection{优化过程}
论文的后面将会看到,原始MKC最优化问题的约束条件数非常之多,难以在有限的时间内求解,因此可以通过割平面法\upcite{kelley1960cutting}求解。通过构造一个可嵌套的、逐渐逼近原始MKC最优化问题的松弛序列,序列中的每个最优化问题都可以看作二阶锥规划(SOCP)\upcite{socp},并能通过凹凸规划(CCCP)\upcite{cccp}来求解。

\section{模型推导}
\subsection{单核最大间隔聚类}
在第三章中已经介绍MMC的核心思想,就是将最大间隔标准从有监督的学习推广到无监督的学习中。在二类聚类情况下,给予数据集:$X=\{\mathbf{x}_1,\cdots,\mathbf{x}_m\}$,MMC的目标是寻找最优的标签集合$\mathbf{y}=\{y_1,\cdots,y_n\}\in\{-1,\,+1\}^n$,使得SVM在数据集$\{(\mathbf{x}_1,y_1),\cdots,(\mathbf{x}_m,y_m)\}$上训练并产生最大间隔。对最优化问题(\ref{equ:MMC-origin})重新公式化描述如下:
\begin{equation}
\begin{split}
\min_{y\in\{\pm1\}^n}\min_{\mathbf{w},b,\mathbf{\xi}} \quad & \frac{1}{2}\mathbf{w}^T\mathbf{w}+\frac{C}{n}\sum^n_{i=1}\xi_i \\
s.t. \quad & \forall i\in\{1,\cdots,m\}: \\
& y_i(\mathbf{w}^T\Phi(\mathbf{x}_i)+b) \ge 1-\xi_i,\xi_i \ge 0,  \\
& -l \le \sum^n_{i=1}y_i \le l.
\label{equ:MKC-singalKernel}
\end{split}
\end{equation}

其中,数据样本$X$被映射到高维特征空间,$\Phi$是线性或非线性特征映射。在支持向量机中,通常是使用其对偶形式进行训练的,借助核方法来隐含使用$\Phi$。$\Phi(\mathbf{x})$可以通过计算相应的核矩阵$K$的乔姆斯基(Cholesky)分解得到,也即$K=\hat{X}\hat{X}^T,\Phi(\mathbf{x}_i)=(\hat{X}_{i,1},\cdots,\hat{X}_{i,n})^T$,或者使用特征分解也能得到相同的结果。

此外,问题(\ref{equ:MKC-singalKernel})中的最后一个约束是类平衡约束,其目的是避免所有的训练样本点都被分配相同的类标签。这里$l > 0$是控制类平衡的常数。

根据最优化问题(\ref{equ:MKC-singalKernel}),单核最大间隔聚类在最大化间隔时同时考虑类标记向量$\mathbf{y}$和分离超平面参数$(\mathbf{w},b)$。未知的二值向量$\mathbf{y}$使最优化问题(\ref{equ:MKC-singalKernel})变成整形规划,比SVM中的二次规划(QP)问题更难解决。然而,参考\upcite{zhao2008efficient},我们能将单核最大间隔聚类问题(\ref{equ:MKC-singalKernel})的等价公式化描述:
\begin{equation}
\begin{split}
\min_{\mathbf{w},b,\mathbf{\xi}}  \quad  & \frac{1}{2}\mathbf{w}^T\mathbf{w}+\frac{C}{n}\sum^n_{i=1}\xi_i \\
s.t. \quad & \forall i\in\{1,\cdots,m\}:  \\
& \left |\mathbf{w}^T\Phi(\mathbf{x}_i)+b \right | \ge 1-\xi_i,\xi_i \ge 0,  \\
& -l \le \sum^n_{i=1} \left [\mathbf{w}^T\Phi(\mathbf{x}_i)+b \right ] \le l.  
\label{equ:MKC-singalKernelRelax}
\end{split}
\end{equation}

其中,标签向量$\mathbf{y}$由$y_i=\mathrm{sign}(\mathbf{w}^T\phi(\mathbf{x}_i)+b)$计算得到,最后一个约束是松弛类平衡约束。与问题(\ref{equ:MKC-singalKernel})相比,问题(\ref{equ:MKC-singalKernelRelax})更加容易处理。

\subsection{多核最大间隔聚类}
由于对某个特定的问题,核函数的适用性是未知的。因此这里利用多个核函数的非负线性组合构造基核,再将这个基核来代替单核最大间隔聚类中的核函数进行训练。详细的说,就是将输入空间中的每个数据样本$\mathbf{x}_i$通过$M$个映射$\Phi_k:\mathbf{x}\mapsto \Phi(\mathbf{x})\in \mathbb{R}^{D_k},k=1,\cdots,M$转换为$M$个特征向量$\Phi_1(\mathbf{x}_i),\cdots,\Phi_M(\mathbf{x}_i)$。这里$D_k$表示第$k$个特征空间的维度。对于每个特征映射来说,都有一个独立的权值向量$\mathbf{w}_k$。从而得到下面的优化问题,当$M=1$时与问题(\ref{equ:MKC-singalKernelRelax})等价。
\begin{equation}
\begin{split}
\min_{\mathbf{\beta},\mathbf{w},b,\mathbf{\xi}} \quad & \frac{1}{2}\sum^M_{k=1}\beta_k\|\mathbf{w}_k\|^2+\frac{C}{n}\sum^n_{i=1}\xi_i \\
s.t. \quad & \forall i \in \{1,\cdots,m\}:  \\
& \left |\sum^M_{k=1}\beta_k\mathbf{w}_k^T\Phi_k(\mathbf{x}_i)+b\right | \ge 1-\xi_i,\xi_i \le 0,   \\
& \forall k \in \{1,\cdots,M\}:\beta_k \ge 0,   \\
& \sum^M_{k=1}\beta_k^p \le 1,   \\
& -l \le \sum^n_{i=1}\left[\sum^M_{k=1}\beta_k\mathbf{w}_k^T\Phi_k(\mathbf{x}_i)+b\right] \le l 
\label{equ:MKC-MultiKernel}
\end{split}
\end{equation}

其中,权值$\beta_k$是用来规则化$M$个输出函数,权值的非负性约束是为了保证核函数之间的线性组合具有凸性,并且能得到基核是半正定的。此外,这里的$p$是一个正整数,这里设定$p=2$,也就是说,使用$l_2$范数对$\mathbf{\beta}=(\beta_1,\cdots,\beta_M)^T$进行规则化。

从问题(\ref{{equ:MKC-MultiKernel}})中很容易看到,由于成对的参数$\beta_k$和$\mathbf{w_k}$使得目标函数、第一个约束以及最后一个约束都是非凸的。因此,这里需要对变量作一些调整:
\begin{equation}
\forall k\in\{1,\cdots,M\}:\mathbf{v}_k=\beta_k\mathbf{w}_k. 
\end{equation}

进过上面的转换,得到与问题(\ref{equ:MKC-MultiKernel})等价的多核MMC公式化描述:
\begin{equation}
\begin{split}
\min_{\mathbf{\beta},\mathbf{v},b,\mathbf{\xi}} \quad & \frac{1}{2}\sum^M_{k=1}\frac{\|\mathbf{v}_k\|^2}{\beta_k}+\frac{C}{n}\sum^n_{i=1}\xi_i\\
s.t. \quad & \forall i \in \{1,\cdots,m\}:   \\
& \left |\sum^M_{k=1}\mathbf{v}_k^T\Phi_k(\mathbf{x}_i)+b\right | \ge 1-\xi_i,\xi_i \le 0,   \\
& \forall k \in \{1,\cdots,M\}:\beta_k \ge 0,   \\
& \sum^M_{k=1}\beta_k^p \le 1,   \\
& -l \le \sum^n_{i=1}\left [ \sum^M_{k=1}\mathbf{v}_k^T\Phi_k(\mathbf{x}_i)+b\right ] \le l  
\label{equ:MKC-MKv}
\end{split}
\end{equation}

其中,$\mathbf{v}=(\mathbf{v}_1,\cdots,\mathbf{v}_M)^T$。现在除了第一个约束条件外,其余的约束条件以及目标函数都具有凸性。

\subsection{割平面算法}
问题(\ref{equ:MKC-MKv})中有$m$个松弛变量$\xi_i$,对应于每个数据样本。接下来首先对问题(\ref{equ:MKC-MKv})重新公式化描述,减少松弛变量的数量。
\begin{theorem} 
\rm 多核MMC可以被等价的公式化描述为(证明见附录B):
\begin{align}
\min_{\mathbf{\beta},\mathbf{v},b,\xi} \quad & \frac{1}{2}\sum^M_{k=1}\frac{\|\mathbf{v}_k\|^2}{\beta_k}+C\xi  \label{equ:MKC-gpm} \\
\nonumber s.t.  \quad & \forall \mathbf{c}\in \{0,1\}^m:   \\
& \frac{1}{m}\sum^m_{i=1}c_i\left |\sum^M_{k=1}\mathbf{v}_k^T\Phi_k(\mathbf{x}_i)+b\right | \ge \frac{1}{m}\sum^m_{i=1}c_i-\xi,  \label{equ:MKC-gpmCon}\\
\nonumber & \forall k \in \{1,\cdots,M\}:\beta_k \ge 0,   \\
\nonumber & \sum^M_{k=1}\beta_k^p \le 1, \xi \ge 0,  \\
\nonumber & -l \le \sum^m_{i=1}\left[\sum^M_{k=1}\mathbf{v}_k^T\Phi_k(\mathbf{x}_i)+b\right] \le l 
\end{align}
\label{theorem:MKC1}
\end{theorem}

在最优化问题(\ref{equ:MKC-gpm})中,松弛变量减少了$m-1$个,所有的非凸约束都共用同一个松弛变量$\xi$,这在很大程度上降低了多核MMC中非凸最优化问题的复杂度。另一方面,方程(\ref{equ:MKC-gpmCon})中约束条件的数量从$m$增加到$2^m$,这种指数级增长十分惊人。然而,割平面算法总能找到整个约束集合的一个小的子集,并且在此约束子集上求解得到的结果仍然能保证足够的精度,从而解决多核MMC问题。详细的说,首先初始化一个空的约束子集$\Omega$,计算在满足约束$\Omega$下问题(\ref{equ:MKC-gpm})的最优解;然后割平面算法会寻找(\ref{equ:MKC-gpmCon})中最违背的约束并添加到约束子集$\Omega$中。通过这种思想,割平面算法构造出一系列逐渐逼近原始多核MMC问题的近似值。如果约束集合(\ref{equ:MKC-gpmCon})中在$\epsilon$范围内没有约束是违背的,那么算法终止。算法1是完整的多核MMC割平面算法。
\begin{table}[htbp]
\centering
 \begin{tabular}{lcl}
  \toprule
  {\CJKfontspec{STHeitiSC-Medium}算法1}\hspace{1em}多核最大间隔聚类的割平面算法 \\
  \midrule
 Input:$M$个特征映射$\Phi_1,\cdots,\Phi_M$,参数$C$,$l$和$\epsilon$,约束子集$\Omega= \phi$ \\
 repeat \\
 \hspace{1em}在当前约束子集$\Omega$下求解问题(\ref{equ:MKC-gpm}),得到$(\mathbf{v},b,\mathbf{\beta})$\\
 \hspace{1em}选择最违背的约束$\mathbf{c}$,令$\Omega = \Omega \cup \{\mathbf{c}\}$\\
 until \\
 \hspace{1em}新选择的约束$\mathbf{c}$违背的程度小于$\epsilon$ \\
  \bottomrule
 \end{tabular}
\end{table}

在上述割平面算法中还存在两个问题:
\begin{enumerate}[fullwidth,itemindent=24pt]
   \item 在给予的约束子集$\Omega$下如何求解问题(\ref{equ:MKC-gpm}){\fangsong ?}
   \item 如何在约束集合(\ref{equ:MKC-gpmCon})中找出最违背的约束{\fangsong ?}
\end{enumerate}

这些将在接下来的两个小节中讨论。

\subsubsection{通过CCCP优化}
在割平面算法的每一轮迭代中,需要在当前的约束子集$\Omega$下求解非凸最优化问题(\ref{equ:MKC-gpm}),从而得到最优的分离超平面。尽管问题(\ref{equ:MKC-gpm})中的目标函数是凸的,但约束并不是凸的,这使得问题很难求解。幸运的是,凹凸规划\upcite{cccp}(CCCP)能解决这类最优化问题。详细的说,问题(\ref{equ:MKC-gpm})中的目标函数是二次的,并且除了第一个约束外所有的约束都是线性的。而且,注意到尽管约束集合(\ref{equ:MKC-gpmCon})中的约束是非凸的,但能够写成两个凸函数的差值:
\begin{equation}
\begin{aligned}
& \forall \mathbf{c} \in \Omega: \\
& \left(\frac{1}{m}\sum^m_{i=1}c_i-\xi\right)-\frac{1}{m}\sum^m_{i=1}c_i\left |\sum^M_{k=1}\mathbf{v}_k^T\Phi_k(\mathbf{x}_i)+b \right | \le 0.
\end{aligned}
\end{equation}

因此,可以通过下面的步骤利用CCCP求解最优化问题(\ref{equ:MKC-gpm})。首先初始化$(\mathbf{v}^{(0)},b^{(0)})$,通过将$\frac{1}{m}\sum^m_{i=1}c_i\left |\sum^M_{k=1}\mathbf{v}_k^T\Phi_k(\mathbf{x}_i)+b \right | $替换为其在$(\mathbf{v}^{(t)},b^{(t)})$处的一阶泰勒展开式,CCCP能够从$(\mathbf{v}^{(t)},b^{(t)})$计算得到$(\mathbf{v}^{(t+1)},b^{(t+1)})$。问题(\ref{equ:MKC-gpm})变成:
\begin{equation}
\begin{split}
\min_{\mathbf{\beta},\mathbf{v},b,\mathbf{\xi}} \quad & \frac{1}{2}\sum^M_{k=1}\frac{\|\mathbf{v}_k\|^2}{\beta_k}+C\xi \\
s.t. \quad & \forall \mathbf{c}\in \Omega: \\
& \frac{1}{m}\sum^m_{i=1}c_i \le \xi + \frac{1}{m}\sum^m_{i=1}c_iz^{(t)}_i \left [\sum^M_{k=1}\mathbf{v}_k^T\Phi_k(\mathbf{x}_i)+b \right ]\\
& \forall k \in \{1,\cdots,M\}:\beta_k \ge 0, \\
& \sum^M_{k=1}\beta_k^2 \le 1, \xi \ge 0,\\
& -l \le \sum^m_{i=1}\left[\sum^M_{k=1}\mathbf{v}_k^T\Phi_k(\mathbf{x}_i)+b\right] \le l
\end{split}
\end{equation}
其中$z^{(t)}_i=\mathrm{sign}(\sum^M_{k=1}\mathbf{v}_k^{(t)T}\Phi_k(\mathbf{x}_i)+b^{(t)})$。引入额外的变量$t_k$作为$\frac{\|\mathbf{v}_k\|^2}{\beta_k}$的上界,我们能将上面的问题公式化描述为二阶锥规划(SOCP)问题:
\begin{equation}
\begin{split}
\min_{\mathbf{\beta},\mathbf{v},b,\mathbf{\xi},\mathbf{t}} \quad & \frac{1}{2}\sum^M_{k=1}t_k+C\xi \\
s.t. \quad & \forall \mathbf{c}\in \Omega: \\
& \frac{1}{m}\sum^m_{i=1}c_i \le \xi + \frac{1}{m}\sum^m_{i=1}c_iz^{(t)}_i \left [\sum^M_{k=1}\mathbf{v}_k^T\Phi_k(\mathbf{x}_i)+b \right ]\\
& \forall k \in \{1,\cdots,M\}: \\
& \left\|\left[\begin{matrix} % or pmatrix or bmatrix or Bmatrix or ...
      2\mathbf{v}_k \\
      t_k-\beta_k \\
   \end{matrix}\right]\right\| \le t_k+\beta_k,
  \beta_k \ge 0, \\
& \sum^M_{k=1}\beta_k^2 \le 1, \xi \ge 0,\\
& -l \le \sum^m_{i=1}\left[\sum^M_{k=1}\mathbf{v}_k^T\Phi_k(\mathbf{x}_i)+b\right] \le l
\label{equ:MKC-SOCP}
\end{split}
\end{equation}

这里我们应用了一个定理,就是形如$\mathbf{s}^T\mathbf{s}\le xy,\ (x,y\in \mathbb{R}_+,\mathbf{s}\in \mathbb{R}^n)$的双曲线约束能被等价地转换为二阶锥约束\upcite{nesterov1994interior,tsang2006efficient}:
\begin{equation}
\left\|\left[\begin{matrix} % or pmatrix or bmatrix or Bmatrix or ...
      2\mathbf{s} \\
      x-y \\
   \end{matrix}\right]\right\| \le x+y,
\end{equation}

由CCCP可知,SOCP问题求得的解$(\mathbf{v},b,\mathbf{\beta},\xi,\mathbf{t})$将作为$(\mathbf{v}^{(t+1)},b^{(t+1)},\mathbf{\beta},\xi,\mathbf{t})$,然后一直迭代直到收敛为止。算法2总结了在满足约束子集$\Omega$下求解问题(\ref{equ:MKC-SOCP})的方法,检查前后两次迭代之间目标函数的差值是否小于$\alpha\%$ (实验中通常将其设置为0.01)作为其终止条件。
\begin{table}[htbp]
\centering
 \begin{tabular}{lcl}
  \toprule
   {\CJKfontspec{STHeitiSC-Medium}算法2}\hspace{1em}满足约束子集$\Omega$条件下通过CCCP求解问题(\ref{equ:MKC-gpm}) \\
  \midrule
 初始化$(\mathbf{v}^{(0)},b^{(0)})$ \\
 repeat \\
 $\quad$求得$(\mathbf{v}^{(t+1)},b^{(t+1)},\mathbf{\beta},\xi,\mathbf{t})$作为SOCP问题(\ref{equ:MKC-SOCP})的解。\\
 $\quad$令$\mathbf{v}=\mathbf{v}^{(t+1)},b=b^{(t+1)},t=t+1$。 \\
 until \\
 $\quad$满足停止准则。\\
  \bottomrule
 \end{tabular}
\end{table}

\newpage

\subsubsection{最违背的约束}
问题(\ref{equ:MKC-gpm})中最违背的约束很容易定义,其约束的可行性由相应的值$\xi$来度量。因此,最违背的约束条件也就是其相应值$\xi$最大。我们用向量$c$来表示约束集合(\ref{equ:MKC-gpmCon})中的每个约束,因此有下面的定理:
\begin{theorem}
\rm 问题(\ref{equ:MKC-gpm})中最违背的约束可以按下面方式计算:
\begin{equation}
c_i=
\begin{cases}
1 & \mathrm{if}\ \left |\sum^M_{k=1}\mathbf{v}_k^T\Phi_k(\mathbf{x}_i)+b \right | < 1 \\
0 & \mathrm{otherwise}
\end{cases}
\label{equ:MVC}
\end{equation}
\label{theorem:MKC2}
\end{theorem}
\hspace{24pt}{\heiti 证明}\hspace{1em}最违背的约束条件也就是其相应值$\xi$最大,为了满足问题(\ref{equ:MKC-gpm})中所有的约束,最优的$\xi$值计算如下:
\begin{align}
\begin{split}
\xi^* & = \sum^m_{i=1}\max_{c_i\in\{0,1\}}\left\{\frac{1}{m}c_i - \frac{1}{m}c_i\left | \sum^M_{k=1}\mathbf{v}_k^T\Phi_k(\mathbf{x}_i)+b \right | \right \} \\
& = \frac{1}{m}\sum^m_{i=1}\max_{c_i\in\{0,1\}}\left\{ c_i \left [ 1 - \left | \sum^M_{k=1}\mathbf{v}_k^T\Phi_k(\mathbf{x}_i)+b \right | \right ] \right \} 
\end{split}
\end{align}

因此,最违背的约束$c$相应的$\xi^*$能够通过等式(\ref{equ:MVC})获得。

割平面算法在迭代时,会选择在当前超平面参数下最违背的约束并将其添加到约束子集$\Omega$,直到在$\epsilon$范围内没有约束是违背的。此外,$\xi$与问题(\ref{equ:MKC-gpm})中的约束集合的可行性有着直接的对应关系,如果点$(\mathbf{v},b,\mathbf{\beta},\xi)$在精度$\epsilon$内满足所有的约束,也就是说:
\begin{align}
\begin{split}
& \forall \mathbf{c} \in \{0,1\}^n: \\
& \frac{1}{m}\sum^m_{i=1}c_i\left | \sum^M_{k=1}\mathbf{v}_k^T\Phi_k(\mathbf{x}_i)+b \right | \ge \frac{1}{m}\sum^m_{i=1}c_i - (\xi + \epsilon)
\label{equ:feasi}
\end{split}
\end{align}

那么点$(\mathbf{v},b,\mathbf{\beta},\xi+\epsilon)$也是可行的。此外,注意到在问题(\ref{equ:MKC-gpm})的目标函数中,松弛变量$\xi$度量了其聚类损失。因此,可以将所有的训练数据都满足不等式(\ref{equ:feasi})作为算法1的终止条件。

\section{本章小结}
MKC对MMC的局限性进行改进,引入多核学习的思想,使用多个核函数的非负线性组合得到的基核进行训练,并使用割平面算法构造一系列逐渐逼近原始多核MMC问题的序列,并且序列中的每个问题都可以通过CCCP进行求解。MKC最终能在训练数据上寻找到最大间隔超平面、最适合的类标记组合以及最优的核函数组合。